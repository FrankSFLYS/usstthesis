\chapter{论文的写作细则}
\label{chap:details}
\section{书写}
\par 正文中的任何部分,如表、图,应限制在版心的范围以内,不要超出版心的范围。
\par 汉字必须使用国家公布的规范字。

\section{标点符号}
\par 标点符号按新闻出版署公布的“标点符号用法”使用。

\section{名词、名称}
\par 科学技术名词术语采用全国自然科学名词审定委员会公布的规范词或国家、部颁标准中规定的名称,尚未统一规定的名词术语,可采用惯用的名称。使用外文缩写代替某一名词术语时,在首次出现处加括号注明其含义。外国人名一般用英文原名,按名前姓后的原则书写。一般很熟知的外国人名(如牛顿、达尔文、马克思等)可按通常标准译法书写中文译名。

\section{量和单位}
\par 量和单位采用中华人民共和国的国家标准GB3100~GB3102-93。非物理量的单位(如件、台、人、元等),可用汉字与符号构成组合形式的单位,如:件/台、元/km等。

\section{数字}
\par 测量统计数据一律用阿拉伯数字。行文叙述个位数时,可用中文数字,如“他发现两颗小行星”、“三力作用于一点”,不宜写成“他发现2颗小行星”、“3力作用于1点”。约数可用中文数字,也可以用阿拉伯数字,如“约一百二十八人",也可写成"约128人”。

\section{标题层次}
\par 标题层次采用采用三级数字编号方法,例如第一级为“第1章”,第二级为“1.1”、“1.2”,第三级为“1.1.1”、“1.1.2”等。层次控制在三级以内,必要时可增设四级。两级之间用下角圆点隔开,每级末尾不加标点。
\par 各层标题均单独占行书写。第一级(章)标题位于新起始页上方正中,第二、三级其标题顶格书写,后空一格书写标题,末尾不加标点。

正文中对总项包括的分项采用(1)、(2)、(3)…的序号,对分项中的小项采用1)、2)、3)...的序号,序号后不再加其他标点,序号前空二格书写。
使用序号的例子:
\begin{enumerate}
    \renewcommand{\arraystretch}{1.0}
    \item 这是第一项。
    \item 这是第二项。
    \begin{enumerate}
        \item 这是第二项的第一小项。
        \item 这是第二项的第二小项。
    \end{enumerate}
    \item 这是第三项。
\end{enumerate}

\section{注释(脚注)}
\par 个别名词或情况需要解释时,可加注说明,注释可用页末注(注文放在加注页的下端)或篇末注(全部注文集中在正文末尾),而不可用行中注(注文夹在正文中间)。

\section{公式}
\par 公式居中书写,建议使用 \verb|equation| 等公式环境,公式较长时应在“$=$”前转行或在“$+$、$-$、$\times$、$\div$”等运算符号处转行,等号或运算符号应在转行后的行首。公式的编号用圆括号括起来放在公式右边行末,公式和编号之间留空。微分算子可使用 \verb|usstthesis| 文档类定义的 \verb|\dif|。例\autoref{equation:1}。
\begin{equation}
    \label{equation:1}
    F(j\omega)=\int_{-\infty}^{\infty}f(t)e^{-j\omega t}\dif t
\end{equation}

\section{表格}
\par 每个表格应有表序和表题,表序和表题应在表格的上方,居中,使用表格环境的 \verb|\caption| 书写,便于添加引用,表格整体置于 \verb|table| 环境中。一般情况下表格采用五号宋体。例\autoref{table:tsp calc}:
\begin{table}[htb]
    \tableCapSet    % 使用此命令调整 caption 间距
    \caption{TSP 问题的计算量(计算速度:$10^8/$s)}
    \label{table:tsp calc}
    \centering
    \zihao{5}
    \begin{tabular}{c|c|c}
        \hlineB{3}  % 线宽为3倍的横线
        城市数目 $n$ & 计算量               & 计算时间                \\
        \hlineB{2}  % 线宽为2倍的横线
        10           & $1.8 \times 10^5$    & $1.8 \times 10^{-3}$ 秒 \\
        \hline
        15           & $4.4 \times 10^{10}$ & $7.3$ 小时              \\
        \hline
        20           & $6.0 \times 10^{16}$ & $19$ 年                 \\
        \hline
        25           & $3.1 \times 10^{23}$ & $1.0 \times 10^7$ 世纪  \\
        \hlineB{3}
    \end{tabular}
\end{table}

\section{插图}
\par 每幅插图应有图序和图题,图序和图题放在图下方居中处,建议使用 \verb|figure| 环境,并且使用 \verb|\includegraphics| 插图。一般情况下插图采用五号宋体。例\autoref{figure:model}:

\begin{figure}[htb]
    \figureCapSet
    \centering
    \includegraphics[width=.8\linewidth]{figure/egf.png}
    \caption{信息传播模型}
    \label{figure:model}
\end{figure}

\section{代码段}
\par 本文档类提供了两种插入代码的方式,其中第一种是使用代码文件插入,使用方法为:\lstinline|\lstinputlisting[caption=xxx]{listing/xxx.c}|。效果如\autoref{code:code.c} 所示。
\codeCapSet
\lstinputlisting[caption=code.c, label=code:code.c]{listing/code.c}

\par 第二种方法是直接在正文当中编辑,使用 \lstinline|code| 和 \lstinline|lstlisting| 环境,使用 \lstinline|captionof{lstlisting}{xxx}| 标记标题。如\autoref{code:main.h} 所示。
\begin{code}
\captionof{lstlisting}{main.h}
\label{code:main.h}
\begin{lstlisting}
int main(int argc, char* argv[]);
\end{lstlisting}
\end{code}

\section{参考文献}

\par 参考文献是毕业设计(论文)中引用文献出处的目录表,一律放在文后。书写格式按国家标准GB7714-87规定。
\par 参考文献按正文中出现的先后统一用阿拉伯数字进行自然编号,序码用方括号括起。且在正文引用处最后一个字的右上角,用方括号标明此序号(如×××\cite{论文},以便查找)。一篇论著在论文中多处被引用时,在参考文献目录表中只应出现一次,序号以第一次出现的位置为准。
\par 具体参考文献的标注格式可参考正规出版的论文杂志(集)。多数论文数据库可以直接生成 bibtex 引用格式。

\section{页面}
\par 从第1章起,奇数页页眉写毕业设计(论文)的题目,偶数页页眉写“上海理工大学本科毕业设计(论文)”,字体为小五号宋体。

\section{页码}
\par 从目录首页到目录末页,在页面底端外侧加注页码,页码为小五号 Times New Roman 小写罗马数字,即 i、ii、iii 等。
\par 从第1章正文起,在页面底端外侧加注页码,页码为小五号 Times New Roman 阿拉伯数字,即1、2、3等。

