\documentclass{usstthesis}

\usstthesisset{%
    session=2015,   % 2015 届
    title=上海理工大学本科毕业设计(论文) \LaTeX 文档类, % 论文中文标题
    titleen=A \LaTeX{} document class for the Final Project/Thesis for Undergraduates of USST,  % 论文英文标题
    institute=光电信息与计算机工程学院, % 学院
    major=计算机科学与技术, % 专业
    name=你的名字,  % 姓名
    number=1234567, % 学号
    mentor=指导教师,    % 指导教师
    date=2019年1月1日   % 完成日期
}

\lstset{language=C}

\renewcommand{\arraystretch}{1.1}   % 表格行高 1.1 倍

\begin{document}
\outputfrontmatter  % 输出封面、承诺书、摘要和目录
% 注意,摘要请在 ./elements/abstract.tex 编辑
    
\mainmatter % 中间内容
% 可以使用分文件编辑,最后 input 每个章节
\chapter{引言}
\label{chap:references}
\par 本文\chapref{chap:contents}介绍了论文每个部分的基本说明和要求,\chapref{chap:formats}介绍了论文格式和字体的要求,\chapref{chap:details}介绍了论文写作的详细规范,其中包含了表格、公式、图片和代码的插入方法,值得一看。 % 论文的内容要求
% 也可以直接在此文件中编写,但是比较乱
\chapter{论文的格式要求}
\par 论文的格式要求包括:纸张大小、纸张方向、页边距、版式、文档网格、字体与字号、段落和间距等\cite{机器学习}。
\par 建议采用 \LaTeX2e 语法和 \verb|xelatex| 引擎编译。
\par 由于论文格式问题非常繁杂,无法将所有设置描述清楚,只能对一些主要的设置做出扼要的说明。一个快捷有效的方法就是把本规范的电子版作为模板。
\par 小贴士:使用 \verb|xelatex| 编译时,可以直接生成双面打印的 PDF 文件。如果需要编译两次生成交叉引用,第一次编译时加入 \verb|--no-pdf| 选项可以加快编译速度。

\lstinputlisting[caption=code.c]{listing/code.c}

\captionof{lstlisting}{main.h}
\begin{lstlisting}
int main(int argc, char* argv[]);
\end{lstlisting}

\chapter{论文的内容要求}
\label{chap:contents}
\par {\bSong 一份完整的毕业设计(论文)包括:标题、基本信息、承诺书、摘要、关键词、目录、正文、参考文献、致谢、附录等。}

\section{标题}
\par 标题包括中文标题和外文标题。

\subsection{中文标题}
\par 中文标题应该简短、明确、有概括性,不宜超过20个汉字。

\subsection{外文标题}
\par 外文标题应该简短、明确、翻译正确。

\section{基本信息}
\par 基本信息包括:学院名称、专业名称、作者姓名、作者学号、指导教师姓名及职称,以及论文完成的日期。

\section{承诺书}
\par 承诺书是论文作者对学术诚信的庄重承诺。本规范提供了上海理工大学本科毕业设计(论文)承诺书的一个规范文本,作者在认真仔细阅读后签上姓名和日期。

\section{摘要}
\par 摘要包括中文摘要和外文摘要两部分。
\par 中外文摘要均包括正文和关键词。

\subsection{摘要正文}
\par 论文摘要简要陈述本科毕业设计(论文)的内容,创新见解和主要论点。中文摘要在500字左右,外文摘要应与中文摘要的内容相符。

\subsection{关键词}
\par 关键词是反映毕业设计(论文)主题内容的名词,是供检索使用的。关键词条应为通用词汇,不得自造关键词。关键词一般为3至5个,按其外延层次,由高至低顺序排列。关键词排在摘要正文部分下方。

\section{目录}
\par 目录按三级标题编写,要求标题层次清晰,并标明页码。

\section{正文}
\par 正文篇幅要求15000字以上(其中,英语、德语专业毕业设计(论文)应不少于5000外文单词,日语专业应不少于8000日语假名,艺术设计类专业不少于3000字。毕业设计(论文)的核心设计、研究篇幅应占全篇幅的三分之二以上。)。内容包括绪论、正文主体与结论。

\subsection{绪论}
\par 绪论是研究工作的概述,内容包括:本课题的意义、目的、研究范围及要达到的技术要求;简述本课题在国内外的发展概况及存在的问题。
\par 绪论一般作为在毕业设计(论文)正文的第1章,并在一章内完成。

\subsection{正文主体}
\par 正文主体是研究工作的详述,内容包括:问题的提出,研究工作的基本前提、假设和条件;模型的建立,实验方案的拟定;基本概念和理论基础;设计计算的主要方法和内容;实验方法、内容及其分析;理论论证及其应用,研究结果,以及对结果的讨论等。
\par 正文主体一般可分为若干章完成。

\subsection{结论}
\par 结论是研究工作的总结,内容包括:对所得结果与已有结果的比较和课题尚存在的问题,以及进一步展开研究的见解与建议。
\par 结论一般作为论文正文的最后一章,并在一章内完成。

\section{参考文献}
\par 参考文献为研究中参考的资料,包括专著、论文、年鉴、网站等。所引用的文献必须是公开发表的与毕业设计(论文)工作直接有关的文献,且经过本人阅读理解。列入的主要文献要求不少于6篇,其中外文文献不少于2篇。

\section{致谢}
\par 毕业设计(论文)是在指导教师的指导下完成,理应致谢。还应对完成论文提供过帮助的其他人员致谢。切忌泛滥和溢美。

\section{附录}
\par 附录是对于一些不宜放在正文中,但又直接反映研究工作的材料(如设计图纸、实验数据、计算机程序等)附于文本末尾。
 % 论文的格式要求
\chapter{论文的格式要求}
\label{chap:formats}
\par 论文的格式要求包括:纸张大小、纸张方向、页边距、版式、文档网格、字体与字号、段落和间距等\cite{机器学习}。
\par 建议采用 \LaTeX2e 语法和 \verb|xelatex| 引擎编译。
\par 由于论文格式问题非常繁杂,无法将所有设置描述清楚,只能对一些主要的设置做出扼要的说明。一个快捷有效的方法就是把本规范的电子版作为模板。
\par 小贴士:使用 \verb|xelatex| 编译时,可以直接生成双面打印的 PDF 文件。如果需要编译两次生成交叉引用,第一次编译时加入 \verb|--no-pdf| 选项可以加快编译速度。

\section{页面设置}
\subsection{纸张}
\par 纸张大小:A4。
\par 纸张方向:纵向。

\subsection{页边距}
\par 页边距;上2.5厘米,下2.5厘米,内侧3厘米,外侧2.5厘米。
\par 页码范围:对称页边距。

\subsection{版式}
\par 节:奇数页。
\par 页眉和页脚:奇偶页不同,距边界:页眉1.5厘米,页脚1.75厘米。

\subsection{文档网格}
\par 网格:无网格。

\subsection{字体}
\par 中文字体:宋体,使用{\bSong 宋体粗体}时,请使用 \verb|{\bSong 要加粗的文字}|。
\par 西文字体:Times New Roman。
\par 字形:常规。
\par 字号:小四。

\subsection{段落}
\par 对齐方式:两端对齐。
\par 首行缩进:2字符。
\par 行距:多倍行距1.25。

\section{封面}
\subsection{标题}
\par 中文标题:二号华文中宋和 Times New Roman 加粗,居中,左、右侧缩进均为4字符。
\par 外文标题:小二号 Times New Roman 加粗,居中,左、右侧缩进均为4字符。

\subsection{基本信息}
\par 基本信息是一个表格,左列为基本信息名称,右列为需要填写的基本信息。
\par 基本信息:四号华文中宋和 Times New Roman 加粗居中。

\section{承诺书}
\par 承诺书:三号华文中宋加粗,居中,段前4行,段后2行。
\par 承诺书文本:小四号宋体和 Times New Roman,首行缩进2字符,1.25倍行距。

\section{摘要}
\par 摘要:三号华文中宋加粗,居中,段前4行,段后2行。
\par 摘要文本:小四号宋体和 Times New Roman,首行缩进2字符,1.25倍行距。
\par 摘要文本结束后空一行。
\par 关键词:小四号宋体加粗顶格,××××:小四号宋体和 Times New Roman,各关键词之间2空格。

\section{ABSTRACT}
\par ABSTRACT:三号Times New Roman加粗,居中,段前4行,段后2行。
\par ABSTRACT文本:小四号Times New Roman,首行缩进2字符,1.25倍行距。
\par ABSTRACT文本结束后空一行。
\par KEY WORDS:小四号Times New Roman加粗顶格,××××:小四号Times New Roman,各关键词之间2空格。

\section{目录}
\par 目录:三号华文中宋加粗,居中,段前4行,段后2行。
\par 以下内容用小四号宋体和 Times New Roman,1.25倍行距:
\par 摘要:加粗,首行缩进2字符,段前0.5行;
\par ABSTRACT:加粗,首行缩进2字符,段前0.5行;
\par 第1章 ××××.........................1:加粗,首行缩进2字符,段前0.5行;
\par \quad 1.1 ××××............................1:加粗,首行缩进3字符;
\par \qquad 1.1.1 ××××.....................1:加粗,首行缩进4字符。

\section{正文}
\par 一级标题:三号华文中宋和 Times New Roman 加粗,居中,段前4行,段后2行。
\par 二级标题:四号宋体和 Times New Roman 加粗,左对齐顶格,段前0.5行,段后0行。
\par 三级标题:小四号宋体和 Times New Roman 加粗,左对齐顶格,段前0.5行,段后0行。
\par 正文文字:小四号宋体和 Times New Roman,首行缩进2字符,1.25倍行距。

\section{参考文献}
\par 参考文献:三号华文中宋加粗,居中,段前4行,段后2行。
\par 参考文献序号用方括号括起。
\par 参考文献序号和内容用五号宋体和 Times New Roman。

\section{致谢}
\par 致谢:三号华文中宋加粗,居中,段前4行,段后2行。
\par 致谢文本:小四号宋体和 Times New Roman,首行缩进2字符,1.25倍行距。
 % 论文的写作细则
% 表格、插图等内容在 elements/chapter3.tex 中有所体现

\backmatter % 结尾内容

\begin{citelist}
    \bibitem{机器学习}刘琴.机器学习[J].武汉工程职业技术学院学报,2001,13(2):41-44.
    \bibitem{论文}中国科学技术信息研究所.2018年度中国科技论文统计与分析[J].科学,2018,70(6):57-59.
\end{citelist}
\chapter{致\quad{}谢}
\par 本论文是在导师的悉心指导下完成的,本文作者在此谨表示衷心的感谢。
\par ××× 老师也对本论文给予了许多宝贵的意见和建议,在此表示深深的谢意。 % 致谢


\end{document}